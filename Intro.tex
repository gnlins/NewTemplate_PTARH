	\section{INTRODUÇÃO}
	Os sistemas de tubulações são dimensionados para atender situações específicas de pressão e vazão, considerando-se que essas condições não vão alterar em função do tempo, ou seja, será mantido um regime permanente na operação dos sistemas. Entretanto, frequentemente, ocorrem situações que alteram essas condições, como, por exemplo, o fechamento de uma válvula ou o corte de energia em uma região ocasionando no desligamento de uma bomba. A previsão destes eventos traz um novo desafio no dimensionamento: os transientes hidráulicos.
	
	Os transientes hidráulicos ocorrem na transição entre dois regimes permanentes. Quando eles acontecem, colocam o sistema em condições mais extremas do que as dimensionadas para um sistema em regime permanente, podendo ocasionar em rompimento de tubulações e, consequentemente, problemas no abastecimento de água, por exemplo. Para se evitar estes problemas operacionais, é necessário compreender como o fenômeno atua e prever as condições que o sistema ficará submetido. Uma das formas de se fazer isso é através de simulações, realizadas por meio da modelagem matemática.
	
	As equações que descrevem o comportamento de um escoamento transitório são diferenciais parciais e, geralmente, não possuem soluções analíticas simples, portanto a opção alternativa é a aplicação de métodos numéricos resolvidos computacionalmente. Diversas abordagens são utilizadas como solução, duas que se destacam são o Método das Características e o Método das Ondas Características. O primeiro apresenta alta acurácia numérica das soluções, entretanto baixa eficiência computacional. O segundo apresenta as qualidades inversas.
	
	Como, apesar de computacionalmente exigentes, os modelos são importantes, a proposta deste trabalho é unir as melhores qualidades dos dois métodos citados, mantendo a acurácia do primeiro, buscar se aproximar da eficiência computacional do segundo por meio da utilização do método das características generalizado.
	
